\documentclass[a4paper]{article}\usepackage{fullpage}\usepackage[german,ngerman]{babel}\usepackage[utf8]{inputenc}\begin{document}\begin{center}{\large{}\textbf{Donnerstag 13:00 - 14:30}}\end{center}\begin{center}\begin{tabular} {|p{5cm}|p{5cm}|p{5cm}|}\hline \textbf{Thema}&\textbf{Betreuer}&\textbf{Raum}\\\hline Spezielle Funktionen und Vektorrechnung&Lilith Diringer&MW 0234\\\hline Einführung ins Differenzieren&Ilja Göthel&MW 1050\\\hline Experimentieren und Auswerten&Ann-Kathrin Raab&MW 0250\\\hline Spezielle Relativitätstheorie&Johannes Rothe&MW 2050\\\hline Thermodynamik 1&Maximilian Marienhagen&MW 1250\\\hline Quanten- und Atomphysik I&Ismail Achmed-Zade&CH 22210\\\hline Elektrische Schaltungen&Christopher Pfeiffer&CH 22209\\\hline \end{tabular}\end{center}\begin{center}{\large{}\textbf{Donnerstag 14:45 - 16:15}}\end{center}\begin{center}\begin{tabular} {|p{5cm}|p{5cm}|p{5cm}|}\hline \textbf{Thema}&\textbf{Betreuer}&\textbf{Raum}\\\hline Experimentieren und Auswerten&Ann-Kathrin Raab&MW 0250\\\hline Näherungsmethoden&Ilja Göthel&MW 0234\\\hline Einführung ins Integrieren&Johannes Rothe&MW 1050\\\hline Klassische Mechanik&Maximilian Marienhagen&MW 1250\\\hline Himmelsmechanik&Lars Dehlwes&MW 2050\\\hline Geometrische Optik&Christopher Pfeiffer&CH 22209\\\hline \end{tabular}\end{center}\begin{center}{\large{}\textbf{Freitag 12:30 - 14:00}}\end{center}\begin{center}\begin{tabular} {|p{5cm}|p{5cm}|p{5cm}|}\hline \textbf{Thema}&\textbf{Betreuer}&\textbf{Raum}\\\hline Spezielle Funktionen und Vektorrechnung&Lilith Diringer&MW 0234\\\hline Einführung ins Differenzieren&Ann-Kathrin Raab&MW 0250\\\hline Rotationsbewegungen&Vincent Grande&MW 1050\\\hline Elektrodynamik 1&Maximilian Keitel&MW 1250\\\hline Aufgabenseminar Wärmelehre&Maximilian Marienhagen&MW 2050\\\hline Kernphysik&Samuel Moll&CH 22209\\\hline Experiment Magnetismus&Lars Dehlwes&Praktikum Magnetismus\\\hline Experiment spezifische Elektronenladung&Felix Wechsler&Praktikum spezifische Elektronenladung\\\hline Experiment Oszilloskop&Christopher Pfeiffer&Praktikum Oszilloskop\\\hline Experiment Brückenschaltung&Martin Großhauser&Praktikum Brückenschaltung\\\hline \end{tabular}\end{center}\begin{center}{\large{}\textbf{Freitag 14:15 - 15:45}}\end{center}\begin{center}\begin{tabular} {|p{5cm}|p{5cm}|p{5cm}|}\hline \textbf{Thema}&\textbf{Betreuer}&\textbf{Raum}\\\hline Gewöhnliche Differentialgleichungen&Maximilian Keitel&MW 0234\\\hline Experimentieren und Auswerten&Ann-Kathrin Raab&MW 0250\\\hline Komplexe Wechselstromrechnung&Vincent Grande&MW 1050\\\hline Spezielle Relativitätstheorie&Johannes Rothe&MW 1250\\\hline Harmonische Schwingungen&Ilja Göthel&MW 2050\\\hline Kernphysik&Samuel Moll&CH 22209\\\hline Experiment Magnetismus&Lars Dehlwes&Praktikum Magnetismus\\\hline Experiment spezifische Elektronenladung&Felix Wechsler&Praktikum spezifische Elektronenladung\\\hline Experiment Oszilloskop&Christopher Pfeiffer&Praktikum Oszilloskop\\\hline Experiment Brückenschaltung&Martin Großhauser&Praktikum Brückenschaltung\\\hline \end{tabular}\end{center}\begin{center}{\large{}\textbf{Samstag 07:00 - 08:30}}\end{center}\begin{center}\begin{tabular} {|p{5cm}|p{5cm}|p{5cm}|}\hline \textbf{Thema}&\textbf{Betreuer}&\textbf{Raum}\\\hline Einführung ins Integrieren&Felix Wechsler&MW 2050\\\hline Experimentieren und Auswerten&Ann-Kathrin Raab&MW 0250\\\hline Thermodynamik 2 - Statistische Physik&Vitaly Andreev&MW 1050\\\hline Elektrodynamik 1&Maximilian Keitel&MW 1250\\\hline Aufgabenseminar klassische Mechanik&Maximilian Marienhagen&MW 0234\\\hline Harmonische Schwingungen&Ilja Göthel&CH 22210\\\hline Experiment Brennstoffzelle&Vincent Grande&Praktikum Brennstoffzelle\\\hline Experiment Millikan-Versuch&Samuel Moll&Praktikum Millikan-Versuch\\\hline Experiment Pohlsches Rad&Eugen Dizer&Praktikum Pohlsches Rad\\\hline Experiment Optische Abbildungen&Lilith Diringer&Praktikum Optische Abbildungen\\\hline \end{tabular}\end{center}\begin{center}{\large{}\textbf{Samstag 08:45 - 10:15}}\end{center}\begin{center}\begin{tabular} {|p{5cm}|p{5cm}|p{5cm}|}\hline \textbf{Thema}&\textbf{Betreuer}&\textbf{Raum}\\\hline Spezielle Relativitätstheorie&Johannes Rothe&MW 0250\\\hline Klassische Mechanik&Maximilian Marienhagen&MW 1050\\\hline Geometrische Optik&Christopher Pfeiffer&MW 0234\\\hline Elektrodynamik 2&Maximilian Keitel&MW 1250\\\hline Quanten- und Atomphysik I&Vitaly Andreev&MW 2050\\\hline Himmelsmechanik&Lars Dehlwes&CH 22209\\\hline Experiment Brennstoffzelle&Vincent Grande&Praktikum Brennstoffzelle\\\hline Experiment Millikan-Versuch&Samuel Moll&Praktikum Millikan-Versuch\\\hline Experiment Pohlsches Rad&Eugen Dizer&Praktikum Pohlsches Rad\\\hline Experiment Optische Abbildungen&Lilith Diringer&Praktikum Optische Abbildungen\\\hline \end{tabular}\end{center}\begin{center}{\large{}\textbf{Samstag 12:30 - 14:00}}\end{center}\begin{center}\begin{tabular} {|p{5cm}|p{5cm}|p{5cm}|}\hline \textbf{Thema}&\textbf{Betreuer}&\textbf{Raum}\\\hline Bestimmung des Brechungskoeffizienten von Wasser&Lilith Diringer&MW 0234\\\hline Gravitationsbeschleunigung&Ann-Kathrin Raab&MW 1050\\\hline Rotationsbewegungen&Vincent Grande&MW 0250\\\hline Thermodynamik 1&Maximilian Marienhagen&MW 1250\\\hline Elektrodynamik 2&Maximilian Keitel&MW 2050\\\hline Harmonische Schwingungen&Ilja Göthel&CH 22209\\\hline Elektrische Schaltungen&Felix Wechsler&CH 27401\\\hline Relativistische Teilchenphysik&Lars Dehlwes&CH 22210\\\hline Quanten- und Atomphysik I&Vitaly Andreev&CH 27402\\\hline \end{tabular}\end{center}\begin{center}{\large{}\textbf{Samstag 14:15 - 15:45}}\end{center}\begin{center}\begin{tabular} {|p{5cm}|p{5cm}|p{5cm}|}\hline \textbf{Thema}&\textbf{Betreuer}&\textbf{Raum}\\\hline Bestimmung des Brechungskoeffizienten von Plexiglas&Lilith Diringer&MW 0234\\\hline Komplexe Wechselstromrechnung&Vincent Grande&MW 0250\\\hline Elektrische Blackboxen&Eugen Dizer&MW 1050\\\hline Gewöhnliche Differentialgleichungen&Maximilian Keitel&MW 1250\\\hline Aufgabenseminar klassische Mechanik&Maximilian Marienhagen&MW 2050\\\hline Himmelsmechanik&Lars Dehlwes&CH 22209\\\hline Wellenoptik&Christopher Pfeiffer&CH 22210\\\hline Quanten- und Atomphysik II&Vitaly Andreev&CH 27402\\\hline \end{tabular}\end{center}\begin{center}{\large{}\textbf{Sonntag 07:00 - 08:30}}\end{center}\begin{center}\begin{tabular} {|p{5cm}|p{5cm}|p{5cm}|}\hline \textbf{Thema}&\textbf{Betreuer}&\textbf{Raum}\\\hline Gravitationsbeschleunigung&Ann-Kathrin Raab&MW 1050\\\hline Aufgabenseminar Elektrodynamik&Maximilian Keitel&MW 0234\\\hline Aufgabenseminar Wärmelehre&Maximilian Marienhagen&CH 22209\\\hline Theoretische Mechanik&Eugen Dizer&CH 22210\\\hline Wellenoptik&Christopher Pfeiffer&MW 2050\\\hline Thermodynamik 2 - Statistische Physik&Vitaly Andreev&MW 1250\\\hline Elektronik&Martin Großhauser&CH 27402\\\hline Spezielle Relativitätstheorie&Johannes Rothe&MW 0250\\\hline \end{tabular}\end{center}\begin{center}{\large{}\textbf{Sonntag 08:45 - 10:15}}\end{center}\begin{center}\begin{tabular} {|p{5cm}|p{5cm}|p{5cm}|}\hline \textbf{Thema}&\textbf{Betreuer}&\textbf{Raum}\\\hline Bestimmung des Brechungskoeffizienten von Plexiglas&Lilith Diringer&MW 0234\\\hline Rotationsbewegungen&Vincent Grande&CH 27402\\\hline Elektrische Blackboxen&Eugen Dizer&MW 1050\\\hline Elektrodynamik 2&Maximilian Keitel&MW 2050\\\hline Relativistische Teilchenphysik&Lars Dehlwes&CH 22210\\\hline Aufgabenseminar Quanten- und Atomphysik und Struktur der Materie&Vitaly Andreev&MW 1250\\\hline Wellenoptik&Christopher Pfeiffer&MW 0250\\\hline Aufgabenseminar klassische Mechanik&Maximilian Marienhagen&CH 22209\\\hline \end{tabular}\end{center}\end{document}