\documentclass[a4paper]{article}\usepackage{fullpage}\begin{document}\begin{center}{\large{}\textbf{Donnerstag 12:30 - 14:00}}\end{center}\begin{center}\begin{tabular} {|p{5cm}|p{5cm}|p{5cm}|}\hline \textbf{Thema}&\textbf{Betreuer}&\textbf{Raum}\\\hline Thermodynamik 1&Georg Krause&MWP, E-Saal\\\hline Einführung ins Differenzieren&Ann-Kathrin Raab&MWP, Mechanik-Gang\\\hline Komplexe Wechselstromrechnung&Hanna Werner&MWP, Optik\\\hline Experimentieren und Auswerten&Herr Dietzel&5.03.201\\\hline Spezielle Funktionen und Vektorrechnung&Arne Wolf&5.03.204\\\hline \end{tabular}\end{center}\begin{center}{\large{}\textbf{Donnerstag 14:15 - 15:45}}\end{center}\begin{center}\begin{tabular} {|p{5cm}|p{5cm}|p{5cm}|}\hline \textbf{Thema}&\textbf{Betreuer}&\textbf{Raum}\\\hline Spezielle Relativitätstheorie&Georg Krause&MWP, E-Saal\\\hline Theoretische Mechanik&Ann-Kathrin Raab&MWP, Mechanik-Gang\\\hline Gewöhliche Differentialgleichungen&Hanna Werner&MWP, Optik\\\hline Looping und Bestimmung der Rollreibung&Herr Dietzel&5.03.201\\\hline \end{tabular}\end{center}\begin{center}{\large{}\textbf{Freitag 12:00 - 14:00}}\end{center}\begin{center}\begin{tabular} {|p{5cm}|p{5cm}|p{5cm}|}\hline \textbf{Thema}&\textbf{Betreuer}&\textbf{Raum}\\\hline Elektrische Schaltungen&Georg Krause&MWP, E-Saal\\\hline Experiment Erdmagnetfeld&Ann-Kathrin Raab&MWP, Mechanik-Gang\\\hline Experiment Kugelfallviskosimeter&Hanna Werner&MWP, Optik\\\hline Experiment Optik&Herr Dietzel&5.03.201\\\hline Experiment Millikanversuch&Arne Wolf&5.03.204\\\hline Experiment Torsionsmodul&Georg Schröter&5.03.203\\\hline Experiment Wärmekapazität bestimmen&Johannes Rothe&5.03.206\\\hline Harmonische Schwingungen&Vincent Stimper&5.03.224\\\hline Quanten- und Atomphysik I&Sebastian Linß&5.03.225\\\hline Einführung ins Integrieren&Klara Knupfer&5.03.229\\\hline Klassische Mechanik&Felix Wechsler&5.03.29\\\hline Fluiddynamik&Maximilian Keitel&5.03.36\\\hline \end{tabular}\end{center}\begin{center}{\large{}\textbf{Freitag 14:15 - 15:45}}\end{center}\begin{center}\begin{tabular} {|p{5cm}|p{5cm}|p{5cm}|}\hline \textbf{Thema}&\textbf{Betreuer}&\textbf{Raum}\\\hline Elektrodynamik 1&Georg Krause&MWP, E-Saal\\\hline \end{tabular}\end{center}\begin{center}{\large{}\textbf{Samstag 06:45 - 08:15}}\end{center}\begin{center}\begin{tabular} {|p{5cm}|p{5cm}|p{5cm}|}\hline \textbf{Thema}&\textbf{Betreuer}&\textbf{Raum}\\\hline Experiment zur Mechanik&Georg Krause&MWP, E-Saal\\\hline Experiment zur Optik&Ann-Kathrin Raab&MWP, Mechanik-Gang\\\hline Experiment zur Elektronik&Hanna Werner&MWP, Optik\\\hline Experiment im Elektronikpraktikum&Herr Dietzel&5.03.201\\\hline Näherungsmethoden&Arne Wolf&5.03.204\\\hline Rotationsbewegungen&Georg Schröter&5.03.203\\\hline Relativistische Teilchenphysik&Johannes Rothe&5.03.206\\\hline \end{tabular}\end{center}\begin{center}{\large{}\textbf{Samstag 08:30 - 10:00}}\end{center}\begin{center}\begin{tabular} {|p{5cm}|p{5cm}|p{5cm}|}\hline \textbf{Thema}&\textbf{Betreuer}&\textbf{Raum}\\\hline Aufgabenseminar SRT&Georg Krause&MWP, E-Saal\\\hline Himmelsmechanik&Ann-Kathrin Raab&MWP, Mechanik-Gang\\\hline Aufgabenseminar klassische Mechanik&Hanna Werner&MWP, Optik\\\hline \end{tabular}\end{center}\begin{center}{\large{}\textbf{Samstag 13:00 - 14:30}}\end{center}\begin{center}\begin{tabular} {|p{5cm}|p{5cm}|p{5cm}|}\hline \textbf{Thema}&\textbf{Betreuer}&\textbf{Raum}\\\hline Aufgabenseminar Quanten- und Atomphysik und Struktur der Materie&Georg Krause&MWP, E-Saal\\\hline Gravitationsbeschleunigung&Ann-Kathrin Raab&MWP, Mechanik-Gang\\\hline Elektronik&Hanna Werner&MWP, Optik\\\hline \end{tabular}\end{center}\begin{center}{\large{}\textbf{Samstag 14:45 - 16:15}}\end{center}\begin{center}\begin{tabular} {|p{5cm}|p{5cm}|p{5cm}|}\hline \textbf{Thema}&\textbf{Betreuer}&\textbf{Raum}\\\hline Aufgabenseminar Wärmelehre&Georg Krause&MWP, E-Saal\\\hline Kernphysik&Ann-Kathrin Raab&MWP, Mechanik-Gang\\\hline Wellenoptik&Hanna Werner&MWP, Optik\\\hline Geometrische Optik&Herr Dietzel&5.03.201\\\hline Thermodynamik 2 - Statistische Physik&Arne Wolf&5.03.204\\\hline Bestimmung des Brechungskoeffizienten von Wasser&Georg Schröter&5.03.203\\\hline \end{tabular}\end{center}\begin{center}{\large{}\textbf{Sonntag 07:15 - 08:45}}\end{center}\begin{center}\begin{tabular} {|p{5cm}|p{5cm}|p{5cm}|}\hline \textbf{Thema}&\textbf{Betreuer}&\textbf{Raum}\\\hline Quanten- und Atomphysik II&Georg Krause&MWP, E-Saal\\\hline Bestimmung des Brechungskoeffizienten von Plexiglas &Ann-Kathrin Raab&MWP, Mechanik-Gang\\\hline \end{tabular}\end{center}\begin{center}{\large{}\textbf{Sonntag 09:00 - 10:30}}\end{center}\begin{center}\begin{tabular} {|p{5cm}|p{5cm}|p{5cm}|}\hline \textbf{Thema}&\textbf{Betreuer}&\textbf{Raum}\\\hline Elektrodynamik 2&Georg Krause&MWP, E-Saal\\\hline Aufgabenseminar Elektrodynamik&Ann-Kathrin Raab&MWP, Mechanik-Gang\\\hline \end{tabular}\end{center}\end{document}